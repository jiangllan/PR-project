% !TeX root = ../pr-project.tex

\chapter{任务简介}

研究生学位论文撰写,除表达形式上需要符合一定的格式要求外,内容方面上也要遵循一些共性原则。

通常研究生学位论文只能有一个主题(不能是几块工作拼凑在一起),该主题应针对某学科领域中的一个具体问题展开深入、系统的研究,并得出有价值的研究结论。
学位论文的研究主题切忌过大,例如,“中国国有企业改制问题研究”这样的研究主题过大,因为“国企改制”涉及的问题范围太广,很难在一本研究生学位论文中完全研究透彻。



\section{商品图像检索任务及主流解决方法}

除国际研究生外,学位论文一律须用汉语书写。
学位论文应当用规范汉字进行撰写,除古汉语研究中涉及的古文字和参考文献中引用的外文文献之外,均采用简体汉字撰写。

国际研究生一般应以中文或英文书写学位论文,格式要求同上。
论文须用中文封面。

研究生学位论文是学术作品,因此其表述要严谨简明,重点突出,专业常识应简写或不写,做到立论正确、数据可靠、说明透彻、推理严谨、文字凝练、层次分明,避免使用文学性质的或带感情色彩的非学术性语言。

论文中如出现一个非通用性的新名词、新术语或新概念,需随即解释清楚。



\section{机器学习方法在商品图像检索中的应用}

论文题目应简明扼要地反映论文工作的主要内容,力求精炼、准确,切忌笼统。
论文题目是对研究对象的准确、具体描述,一般要在一定程度上体现研究结论,因此,论文题目不仅应告诉读者这本论文研究了什么问题,更要告诉读者这个研究得出的结论。
例如:“在事实与虚构之间:梅乐、卡彭特、沃尔夫的新闻观”就比“三个美国作家的新闻观研究”更专业、更准确。



\section{本文的主要贡献}

论文摘要是对论文研究内容的高度概括,应具有独立性和自含性,即应是 一篇简短但意义完整的文章。
通过阅读论文摘要,读者应该能够对论文的研究 方法及结论有一个整体性的了解,因此摘要的写法应力求精确简明。
论文摘要 应包括对问题及研究目的的描述、对使用的方法和研究过程进行的简要介绍、 对研究结论的高度凝练等,重点是结果和结论。

论文摘要切忌写成全文的提纲,尤其要避免“第 1 章……;第 2 章……;……”这样的陈述方式。